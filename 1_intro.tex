\vspace{-0.1in}
\section{Introduction} \label{sec:intro}
\vspace{-0.1in}
There is rapidly growing interest in singular value decomposition (SVD) in various fields,
including signal processing, computer vision, information retrieval, machine learning and information theory.
Yet, most algorithms for computing SVD are designed to solve problems with only small-scale data in acceptable time and space requirements due to their algorithmic complexities.
With the advent of the ``big data'' era, these algorithms are no longer adequate in terms of computational complexity and required memory space.

%A typical SVD algorithm can be broken down into two steps \cite{65SIAM}.
%The first step is to reduce the initial matrix to bidiagonal.
%The second step is to diagonalize the resulting matrix.
%Most of the literature focuses on the second step as algorithms in second step typically use iterative approaches \cite{58iter1,90iter2,65iter3} and thus take the bulk of the computation time depending on the accuracy requirement.

QR iteration is an algorithm that is widely used in sofeware application.
Jacobi algorithm is the one with the highest accuracy in practice \cite{97bookalgebra}.
However, the cost of $O(n^3)$ complexity with a big constant causes these algorithms very slow.
Divide-and-conquer (DC) is assumed to be the fastest algorithm for large matrices \cite{94DCSVD}.
It takes $O(n^{2.3})$ flops on average \cite{97bookalgebra}. 
But the major drawback of DC is the relatively low accuracy of the singular values when merging, let alone singular vectors. 
In summary, the prior works on SVD computations are either time-consuming or inaccurate.

In addition, all algorithms discussed above have three common disadvantages:
(1) heavy data dependence makes SVD algorithm not suitable for parallelization and extension for other architectures;
(2) large memory space required for temporary variables dramatically limits the capability for computing singular values for very large matrices, and
(3) these algorithms are unable to calculate the subsets directly, while many SVD applications, such as principal component analysis (PCA), need only a small subset of the singular values and vectors.

In this paper, we present a new SVD approach called ``Bisection and Twisted'' (BT) algorithm and implement it on GPU. 
Compared to other algorithms, BT approach only requires $O(n^2)$ to complete the whole SVD\cite{05UCB,09NLAAtwisted}.
Most importantly, there are three salient features that make BT algorithm attractive for large matrices. 
First, the data dependency is weak in the BT algorithm, which make it an excellent candidate for parallel computing. 
Second, the algorithm can obtain a subset of $k$ singular values and its corresponding vectors in $O(kn)$ time. 
This is particularly useful for applications that do not require a complete SVD.
Third, the algorithm needs only $O(kn)$ memory space to store temporary variables, which is important for extending to large scale matrices in big data applications on memory constrained platforms.
We design a multi-GPU version that can scale well with the matrix size. To the best of our knowledge, we are the first to achieve bidiagonal SVD on a 1 million by 1 million matrix using just two GPUs.
We also perform in-depth analysis on the GPU kernels for singular vector calculation, and present multiple optimization methods to further improve the SVD performance on GPUs.

The rest of the paper is organized as follows.
The Bisection and Twisted algorithm is given in Section \ref{sec:algorithm}.
Section \ref{sec:implementation} describes the implementation of the BT algorithm on GPUs, as well as the GPU specific optimizations.
Section \ref{sec:results} presents the experimental results and profiling analysis of GPU kernels.
Section \ref{sec:related} discusses the related work.
Conclusion and future work are in Section \ref{sec:conclusion}.

